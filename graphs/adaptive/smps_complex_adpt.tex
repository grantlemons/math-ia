\documentclass[../../../IA.tex]{subfiles}

\begin{document}

\newcommand{\sumdiv}{\exampleLowSubintervals}
\newcommand{\leftbound}{-4}
\newcommand{\rightbound}{4}

\begin{figure}[ht]
    \centering
    \resizebox{.7\textwidth}{!}{
        \begin{tikzpicture}[extended line/.style={shorten >=-#1,shorten <=-#1}, spy using outlines={circle, magnification=9, size=2cm, connect spies}]
            \def\arr{-4.0, -3.0, -2.75, -2.5, -2.375, -2.25, -2.125, -2.0, -1.875, -1.75, -1.625, -1.5, -1.375, -1.25, -1.0, -0.75, -0.625, -0.5625, -0.5, -0.4375, -0.375, -0.3125, -0.25, -0.21875, -0.1875, -0.15625, -0.125, -0.09375, -0.0625, -0.03125, 0.0, 0.03125, 0.0625, 0.09375, 0.125, 0.15625, 0.1875, 0.21875, 0.25, 0.3125, 0.375, 0.4375, 0.5, 0.5625, 0.625, 0.75, 1.0, 1.25, 1.375, 1.5, 1.625, 1.75, 1.875, 2.0, 2.125, 2.25, 2.375, 2.5, 2.75, 3.0, 4.0}
            \pgfmathparse{dim(\arr)-1}
            \foreach \x [count=\i] in {0, ..., \pgfmathresult}
            {
                \pgfmathsetmacro\a{\arr[\i]}
                \pgfmathsetmacro\m{(\arr[i]+\arr[\i+1])/2}
                \pgfmathsetmacro\b{\arr[\i+1]}

                \pgfmathsetmacro\fA{\normalEq{\a}}
                \pgfmathsetmacro\fM{\normalEq{\m}}
                \pgfmathsetmacro\fB{\normalEq{\b}}

                \draw [gray, thin, pattern=north west lines, pattern color=gray, samples=100, domain=\a:\b, variable=\x] (\a, 0) -- plot(\x,{((\fA*(\x-\m)*(\x-\b))/((\a-\m)*(\a-\b))) + ((\fM*(\x-\a)*(\x-\b))/((\m-\a)*(\m-\b))) + ((\fB*(\x-\a)*(\x-\m))/((\b-\a)*(\b-\m)))}) -- (\b, 0) -- cycle;
                \draw [gray, thin] (\a, \fA) -- (\a, 0) -- (\b, 0) -- (\b, \fB);
            }

            \draw[black, thick] ({\leftbound-.1}, -0.02) -- ({\rightbound+.1}, -0.02) node[black, align=center, left, yshift=-0.13cm, xshift=-4.3cm] {\scriptsize \(x\)};
            \graphNormal

            % \pgfmathsetmacro\SX{5.5}
            % \coordinate (S) at (\SX, {(((\SX+(2/3))^3 + 2*(\SX)^2)/3) - 0.03});
            % \coordinate (SA) at ({\SX+.25}, -1.2);
            % \spy [darkgray] on (S) in node [right] at (SA);
        \end{tikzpicture}
    }
    \vspace{-.4cm}
    \caption{A sketch demonstrating adaptive quadrature.}
    \label{fig:smps_rule_complex_adpt}
\end{figure}
\end{document}
