\documentclass[../../../IA.tex]{subfiles}

\begin{document}
\newcommand{\sumdiv}{\exampleLowSubintervals}
\newcommand{\leftbound}{-4}
\newcommand{\rightbound}{4}

\begin{figure}[H]
    \centering
    \resizebox{.55\textwidth}{!}{
        \begin{tikzpicture}[extended line/.style={shorten >=-#1,shorten <=-#1}]
            \def\arr{-4.0/1.600, -2.0/-1.620, -1.0/-3.989, 0.0/3.989, 1.0/1.620, 2.0/-1.600}
            \foreach \x/\v in \arr
            {   
                \pgfmathsetmacro\fA{\normalEq{0}}
                \draw[gray, thin] (\x, 0) -- (\x, \fA) node[xshift=.6cm, yshift=.26cm, rotate=60]{\small \v};
            }
            \draw[gray, thin] (4, 0) -- (4, {\normalEq{0}});

            \draw[dashed, lightgray, thin] (-1.0, {\DerDerNormalEq{-1}}) -- (-1.0, {\DerDerNormalEq{0}}) -- (0, {\DerDerNormalEq{0}});
            \node at (0.0, {\DerDerNormalEq{0}}) [darkgray, circle, fill, inner sep=1.5pt]{};
            \node at (-1.0, {\DerDerNormalEq{-1}}) [darkgray, circle, fill, inner sep=1.5pt]{};

            \draw[black, thick] ({\leftbound-.1}, 0) -- ({\rightbound+.1}, 0) node[black, align=center, left, yshift=-0.2cm, xshift=-7.5cm] {\small \(x\)};
            \graphNormal
            \graphDerDerNormal
        \end{tikzpicture}
    }
    \vspace{-.2cm}
    \caption{A sketch demonstrating the subdivisions during the second step of my method of adaptive quadrature}
    \label{fig:my_adpt_steptwo}
\end{figure}
\end{document}
