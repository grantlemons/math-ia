\documentclass[../../../IA.tex]{subfiles}

\begin{document}

\newcommand{\sumdiv}{\exampleHighSubintervals}
\newcommand{\leftbound}{-2.25}
\newcommand{\rightbound}{2.25}

\begin{figure}[ht]
    \centering
    \resizebox{.7\textwidth}{!}{
        \begin{tikzpicture}[extended line/.style={shorten >=-#1,shorten <=-#1}, spy using outlines={circle, magnification=9, size=2cm, connect spies}]
            \pgfmathparse{int(\sumdiv-1)}
            \foreach \i in {0, ..., \pgfmathresult}
            {
                \pgfmathsetmacro\a{\leftbound + (((\rightbound - \leftbound)/\sumdiv)*(\i))}
                \pgfmathsetmacro\m{\leftbound + (((\rightbound - \leftbound)/\sumdiv)*(\i+.5))}
                \pgfmathsetmacro\b{\leftbound + (((\rightbound - \leftbound)/\sumdiv)*(\i+1))}

                \pgfmathsetmacro\fA{\complexEq{\a}}
                \pgfmathsetmacro\fM{\complexEq{\m}}
                \pgfmathsetmacro\fB{\complexEq{\b}}

		\fill [gray, thin, pattern=north west lines, pattern color=gray, samples=100, domain=\a:\b, variable=\x] (\a, 0) -- plot(\x,{((\fA*(\x-\m)*(\x-\b))/((\a-\m)*(\a-\b))) + ((\fM*(\x-\a)*(\x-\b))/((\m-\a)*(\m-\b))) + ((\fB*(\x-\a)*(\x-\m))/((\b-\a)*(\b-\m)))}) -- (\b, 0) -- cycle;
                \draw [gray, thin] (\a, \fA) -- (\a, 0) -- (\b, 0) -- (\b, \fB);
            }

	    \draw[black, thick] ({\leftbound-.1}, -0.02) -- ({\rightbound+.1}, -0.02) node[black, align=center, left, yshift=-0.13cm, xshift=-4.3cm] {\scriptsize \(x\)};
            \graphComplex

            % \pgfmathsetmacro\SX{5.5}
            % \coordinate (S) at (\SX, {(((\SX+(2/3))^3 + 2*(\SX)^2)/3) - 0.03});
            % \coordinate (SA) at ({\SX+.25}, -1.2);
            % \spy [darkgray] on (S) in node [right] at (SA);
        \end{tikzpicture}
    }
    \vspace{-.4cm}
    \caption{A sketch showing the effectiveness of the composite Simpson's 1/3 rule on a complex function with a high subinterval count.}
    \label{fig:smps_rule_complex_high}
\end{figure}
\end{document}
