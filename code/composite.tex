\documentclass[../IA.tex]{subfiles}

\begin{document}
\begin{listing}[!ht]
\centering
\snippet{A function in the program used to calculate the Simpson's 1/3 Rule and Trapezoid Rule for each subinterval of the function \(f(x)\). Written in Rust}
\caption{A function in the program used to calculate the Simpson's 1/3 Rule and Trapezoid Rule for each subinterval of the function \(f(x)\). Written in Rust}
\label{lst:rust_composite}
\begin{minted}[
frame=lines,
framesep=2mm,
baselinestretch=1,
fontsize=\scriptsize,
]{rust}
pub fn composite(mode: Mode, subints: u16, a: f32, b: f32, xcoords: &mut Vec<f32>) -> f32 {
    // Calculate subinterval width
    let subinterval_width = f32::abs(b - a) / subints as f32;

    // Declare and initialize sum variable
    let mut simpsons_sum: f32 = 0.0;

    // Loop through each subinterval
    for i in 0..subints {
        // Calculate subinterval bounds
        let left = a + ((i as f32) * subinterval_width);
        let right = a + ((i + 1) as f32 * subinterval_width);

        // Add left subinterval bound to vector for output
        xcoords.push(left);

        // Store outputs of respective rules
        let mut simpsons_value: Option<f32> = None;

        // Check the conditions for division dependent on mode parameter
        let condition = match mode {
            Mode::Established => {
                simpsons_value = Some(simpsons_rule(left, right));
                f32::abs(trapezoid_rule(left, right) - simpsons_value.unwrap())
                    > mode.threshold()
            }
            Mode::Complexity => complexity(left, right) > mode.threshold(),
            Mode::Simple => false,
        };

        // Recursively divide if above condition met, otherwise use simpsons rule
        simpsons_value = if condition {
            Some(composite(mode, 2, left, right, xcoords))
        } else if simpsons_value.is_none() {
            Some(simpsons_rule(left, right))
        } else {
            simpsons_value
        };

        // Add values of local variables to sum variable
        simpsons_sum += simpsons_value.unwrap();
    }

    // Return the two sums
    simpsons_sum
}
\end{minted}
\end{listing}
\end{document}
