\documentclass[../IA.tex]{subfiles}

\begin{document}
\begin{listing}[H]
\centering
\snippet{A function in the Rust program used to calculate the Simpson's 1/3 Rule and Trapezoid Rule for each subinterval of the function \(f(x)\)}
\caption{A function in the Rust program used to calculate the Simpson's 1/3 Rule and Trapezoid Rule for each subinterval of the function \(f(x)\)}
\label{lst:rust_composite_complexity}
\begin{minted}[
frame=lines,
framesep=2mm,
baselinestretch=1,
fontsize=\footnotesize,
]{rust}
pub fn composite(
    mode: Mode,
    subints: u16,
    a: f32,
    b: f32,
    xcoords: &mut Vec<f32>,
) -> (f32, f32) {
    // Calculate subinterval width
    let subinterval_width = f32::abs(b - a) / subints as f32;

    // Declare and Initialize sum variables
    let mut t_sum: f32 = 0.0;
    let mut s_sum: f32 = 0.0;

    // Loop through each subinterval
    for i in 0..subints {
        // Calculate subinterval bounds
        let left = a + ((i as f32) * subinterval_width);
        let right = a + ((i + 1) as f32 * subinterval_width);

        // Add left subinterval bound to vector for output
        xcoords.push(left);

        // Store outputs of respective rules
        let mut trap_val = trapezoid_rule(left, right);
        let mut smps_val = simpsons_rule(left, right);

        // Check the conditions for division dependent on mode parameter
        let condition = match mode {
            Mode::Established => f32::abs(trap_val - smps_val) > mode.threshold(),
            Mode::Complexity => complexity(left, right) > mode.threshold(),
            Mode::Simple => false,
        };

        // Recursively divide if above condition met
        if condition {
            (trap_val, smps_val) = composite(mode, 2, left, right, xcoords);
        }

        // Add values of local variables to sum variables
        t_sum += trap_val;
        s_sum += smps_val;
    }

    // Return the two sums
    (t_sum, s_sum)
}
\end{minted}
\end{listing}
\end{document}
